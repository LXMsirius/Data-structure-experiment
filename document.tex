\documentclass{ctexart}
\usepackage{listings}
\usepackage[a4page,hmargin = { 3cm,
	.8in } , height
= 10in]{geometry}

\title{\huge 数据结构实验二}
\author{姓名学号成绩}
\date{}
\begin{document}
\maketitle


\section{实验目的}

\section{实验内容}

\section{数据结构及算法描述}

\section{详细设计}

\section{程序代码}












\begin{lstlisting}[language={c}]
\tiny
#include<stdio.h>
#include<stdlib.h>

typedef struct Num_score//定义了一个结构体用于储存每个学生的学号、各科成绩、总分、加权平均分,并取了一个别名student
{
	int Num;
	int score[3];
	int sum;
	float aver;
} student;

void Input_score(int *p,student *stu){//读入已有学生信息
	int i;
	for(i=0;i<*p;i++){
		printf("请依次输入第%d个学生的学号,语文,数学,英语成绩(用空格分开):",i+1);
		scanf("%d %d %d %d",&(stu+i)->Num,&(stu+i)->score[0],&(stu+i)->score[1],&(stu+i)->score[2]);
		while ((stu+i)->score[0]>100||(stu+i)->score[0]<0||(stu+i)->score[1]>100||(stu+i)->score[1]<0||(stu+i)->score[2]>100||(stu+i)->score[2]<0)
		{
			printf("请重新输入第%d个学生的信息:",i+1);
			scanf("%d %d %d",&(stu+i)->score[0],&(stu+i)->score[1],&(stu+i)->score[2]);
		}
		stu[i].sum=stu[i].score[0]+stu[i].score[1]+stu[i].score[2];
		stu[i].aver=stu[i].score[0]*0.3+stu[i].score[1]*0.5+stu[i].score[2]*0.2;
	}
}

void Output_Grade(int *p,student *stu){//输出已存入的学生信息
	int i;
	printf("各学生成绩如下:\n");
	for ( i = 0; i < *p; i++)
	{
		printf("第个%d学生的信息为:\t学号%d\t语文%d\t数学%d\t英语%d\t总分%d\t加权平均分%.2f\t\n",i+1,stu[i].Num,stu[i].score[0],stu[i].score[1],stu[i].score[2],stu[i].sum,stu[i].aver);
	}
}

student* Add_score(int *p,student *stu){//添加学生信息
	// student *p;
	stu=(student*)realloc(stu,(*p+1)*sizeof(student));
	if(!stu) exit(0);
	printf("请依次输入新加入学生成绩:\n");
	scanf("%d %d %d %d",&stu[*p].Num,&stu[*p].score[0],&stu[*p].score[1],&stu[*p].score[2]);
	while (stu[*p].score[0]<0||stu[*p].score[0]>100||stu[*p].score[1]<0||stu[*p].score[1]>100||stu[*p].score[2]<0||stu[*p].score[2]>100)
	{
		printf("请重新依次输入新加入学生成绩:\n");
		scanf("%d %d %d %d",&stu[*p].score[0],&stu[*p].score[1],&stu[*p].score[2]);
	}
	stu[*p].sum=stu[*p].score[0]+stu[*p].score[1]+stu[*p].score[2];
	stu[*p].aver=stu[*p].score[0]*0.3+stu[*p].score[1]*0.5+stu[*p].score[2]*0.2;
	*p++;
	Output_Grade(p,stu);
	// return stu;
}

void Delete_score_Num(int *p,student *stu){
	int i,j,Num;
	printf("请输入要删除学生信息的学号:");
	scanf("%d",&Num);
	for(i=0;i<*p;i++){
		if(stu[i].Num==Num){
			for(j=i;j<*p;j++) stu[j]=stu[j+1];
			*p-=1;i--;
		}
	}
	Output_Grade(p,stu);
}

void Sort_Score_Num(int *p,student *stu){//根据学号升序排列
	int i,j,k;
	student temp;
	for(i=0;i<*p-1;i++){
		k=i;
		for(j=i+1;j<*p;j++){
			if (stu[k].Num>stu[j].Num) k=j;
		}
		temp=stu[i];stu[i]=stu[k];stu[k]=temp;
	}
	printf("根据学号排序如下:\n");
	Output_Grade(p,stu);
}

void Sort_Score_Sum(int *p,student *stu){//根据总分升序排列
	int i,j,k;
	student temp;
	for(i=0;i<*p-1;i++){
		k=i;
		for(j=i;j<*p;j++){
			if(stu[k].sum>stu[j].sum) k=j;
		}
		temp=stu[i];stu[i]=stu[k];stu[k]=temp;
	}
	printf("根据总成绩排序如下:\n");
	Output_Grade(p,stu);
}

int main(){
	int n,Num,c,*p;
	p=&n;
	student *stu;
	printf("请输入数组大小n(1-100):");
	scanf("%d",p);
	while (n<0||n>100)
	{
		printf("请重新输入数组大小(1-100):");
		scanf("%d",p);
	}
	stu=(student*)malloc(n*sizeof(student));
	if(!stu) exit(0);
	Input_score(p,stu);
	// Sort_Score_Num(n,stu);
	Output_Grade(p,stu);
	printf("执行增加学生信息请输入1,按学号排序请输入2,按总成绩排序输入3,删除学生信息请输入4,退出请输入0:");//再初始化学生信息后,选择将要进行的操作
	scanf("%d",&c);
	while (c)
	{
		if(c==1)
		Add_score(p,stu);
		else if (c==2)
		Sort_Score_Num(p,stu);
		else if(c==3)
		Sort_Score_Sum(p,stu);
		else if(c==4)
		Delete_score_Num(p,stu);
		else printf("输入有误,请重新输入!\n");
		printf("执行增加学生信息请输入1,按学号排序请输入2,按总成绩排序输入3,删除学生信息请输入4,退出请输入0:");
		scanf("%d",&c);
	}
}
\end{lstlisting}

\section{测试和结果}
\section{用户手册}
\end{document}